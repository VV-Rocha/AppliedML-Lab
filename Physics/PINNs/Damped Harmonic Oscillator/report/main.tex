\documentclass[letterpaper,12pt]{article}
\usepackage[utf8]{inputenc}

\title{PINNs - Damped Harmonic Oscillator}
\author{V. Rocha}
\date{\vspace{-0ex}}
%%%%%%%%%%%%%%%%%%%%%%%%%%%%%%%%%%%%%%%%%%%%%%%%%%%%%%%%%%%%
%%%%%%%%%%%%%%%	Packages 	%%%%%%%%%%%%%%%%%%%%%%%%%%%%%%%
%%%%%%%%%%%%%%%%%%%%%%%%%%%%%%%%%%%%%%%%%%%%%%%%%%%%%%%%%%%%
\usepackage[english]{babel}
\usepackage{tabularx}
\usepackage{amsmath, amssymb}
\usepackage{graphicx}
\usepackage[margin=1in,letterpaper]{geometry}
\usepackage[final]{hyperref}
\hypersetup{
			colorlinks=true,
			linkcolor=blue,
			citecolor=blue,
			filecolor=magenta,
			urlcolor=blue
}
\usepackage{amsfonts}
\usepackage[]{cite}
\usepackage{physics}
\usepackage{setspace}
\onehalfspacing
\usepackage{comment}
\usepackage[title]{appendix}
\usepackage{rotating}
\usepackage{tikz}

\begin{document}

\maketitle

\begin{abstract}
    
    Short abstract.

\end{abstract}

\section{Introduction}

	The one-dimensional damped harmonic oscillator is described by the following PDE
\begin{equation}
	\ddot{x} + \gamma \dot{x} + \kappa x = 0,
\label{eq: pde damped harmonic}
\end{equation}
	where $\kappa$ is the spring constant and $\gamma$ is a dampening coefficient.


\appendix

\section{Runge-Kutta 4}

	Considering the PDE in equation~\ref{eq: pde damped harmonic} the Runge-Kutta 4 (RK4) method follows the framework
\begin{align}
	\dot{x} & = \dot{v}  \\
	\dot{v} & = -\gamma \dot{x} - \kappa x
\end{align}

\bibliographystyle{unsrt}
\bibliography{mybib}

\end{document}